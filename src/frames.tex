\begin{frame}
  \titlepage
  \hypertarget{titlePage}{}
\end{frame}

\begin{frame}
  \frametitle{Intro}
  \begin{itemize}
    \item ``regular'' indexes map \emph{columns} to \emph{pkeys}
      \pause
    \item GIN indexes map \textbf{\emph{elements inside container columns}} to
      \emph{pkeys}
      \pause
    \item Upstream postgres supports GIN indexes on types
      \begin{itemize}
        \item \sqlinline{tsvector}
        \item \sqlinline{anyarray}
        \item \sqlinline{jsonb}
      \end{itemize}
      \pause
    \item \textinline{pg_trgm} extension for trigram matching supports GIN
      index on type \sqlinline{text}
      \pause
    \item Yugabyte will support GIN indexes on the same types, but
      prioritization is in the order above
  \end{itemize}
\end{frame}

\begin{frame}
  \frametitle{DocDB schema}
  \sqlinline{CREATE TABLE arrs (pk char, a int[]);}

  creates a DocDB table with schema
  \begin{itemize}
    \item \sqlinline{pk char} $\rightarrow$ \sqlinline{a int[]}
  \end{itemize}
  \pause

  Assuming it were possible to create index on int array,

  \sqlinline{CREATE INDEX ON arrs (a);}

  would create a DocDB index with schema
  \begin{itemize}
    \item \sqlinline{a int[]} $\rightarrow$ \sqlinline{pk char}
  \end{itemize}
  \pause

  \sqlinline{CREATE INDEX ON arrs USING gin (a);}

  creates a DocDB index with schema
  \begin{itemize}
    \item \sqlinline{a}$'$\sqlinline{ int} $\rightarrow$ \sqlinline{pk char}
  \end{itemize}

\end{frame}

\begin{frame}
  \frametitle{Write path}
  \sqlinline{INSERT INTO arrs VALUES ('i', ARRAY[1, 3, 5, 3]);}
  \pause

  \begin{itemize}
    \item insert to the main table
      \pause
      \begin{itemize}
        \item \sqlinline{'i'} $\mapsto$ \sqlinline{ARRAY[1, 3, 5, 3]}
      \end{itemize}
      \pause
    \item insert to the secondary index
      \pause
      \begin{itemize}
        \item extract deduped scan entries: \sqlinline{1}, \sqlinline{3},
          \sqlinline{5}
          \pause
        \item write records \emph{entry} $\mapsto$ \emph{pk}:
          \pause
          \begin{itemize}
            \item \sqlinline{1} $\mapsto$ \sqlinline{'i'}
              \pause
            \item \sqlinline{3} $\mapsto$ \sqlinline{'i'}
              \pause
            \item \sqlinline{5} $\mapsto$ \sqlinline{'i'}
          \end{itemize}
      \end{itemize}
  \end{itemize}
\end{frame}

\begin{frame}
  \frametitle{More inserts}
  For the sake of future examples, insert more rows:

  \sqlinline{-- previous: INSERT INTO arrs VALUES ('i', ARRAY[1, 3, 5, 3]);}

  \sqlinline{INSERT INTO arrs VALUES ('j', ARRAY[1, 3, 2]);}

  \sqlinline{INSERT INTO arrs VALUES ('k', ARRAY[5, 3]);}
  \pause

  What would the index look like after these inserts?
  \pause

  \begin{itemize}
    \item \sqlinline{1} $\mapsto$ \sqlinline{'i'}
    \item \sqlinline{1} $\mapsto$ \sqlinline{'j'}
    \item \sqlinline{2} $\mapsto$ \sqlinline{'j'}
    \item \sqlinline{3} $\mapsto$ \sqlinline{'i'}
    \item \sqlinline{3} $\mapsto$ \sqlinline{'j'}
    \item \sqlinline{3} $\mapsto$ \sqlinline{'k'}
    \item \sqlinline{5} $\mapsto$ \sqlinline{'i'}
    \item \sqlinline{5} $\mapsto$ \sqlinline{'k'}
  \end{itemize}
\end{frame}

\begin{frame}
  \frametitle{Read path}
  Select rows whose array column intersects with \sqlinline{ARRAY[5]}:

  \sqlinline{SELECT * FROM arrs WHERE a && ARRAY[5];}
  \pause

  \emph{Note: this will be an index scan (not index only scan).}
  \pause

  \begin{itemize}
    \item extract scan entries: \sqlinline{5}
      \pause
    \item scan index
      \begin{itemize}
        \item \emph{Note: pggate scans have ``binds'' and ``targets'', and the
          requests to DocDB behave like ``\sqlinline{SELECT <targets> FROM
          <rel> WHERE <binds>}''}
          \pause
        \item bind: \textinline{a}$'$ $=$ \sqlinline{5}
          \pause
        \item target: \cinline{YBIdxBaseTupleIdAttributeNumber}
          \pause
        \item get: \sqlinline{'i'}, \sqlinline{'k'}
      \end{itemize}
  \end{itemize}
\end{frame}

\begin{frame}
  \frametitle{Read path (continued)}
  \begin{itemize}
    \item scan main table
      \begin{itemize}
        \item bind: \textinline{pk} $=$ \sqlinline{'i'}
          \pause
        \item target: \cinline{YBTupleIdAttributeNumber} (not really needed)
        \item target: \textinline{pk} (for \sqlinline{SELECT *})
        \item target: \textinline{a} (for \sqlinline{SELECT *})
          \pause
        \item get: \{\sqlinline{'i'}, \sqlinline{ARRAY[1, 3, 5, 3]}\}
      \end{itemize}
      \pause
    \item scan main table
      \begin{itemize}
        \item bind: \textinline{pk} $=$ \sqlinline{'k'}
          \pause
        \item target: (same as above)
          \pause
        \item get: \{\sqlinline{'k'}, \sqlinline{ARRAY[5, 3]}\}
      \end{itemize}
      \pause
    \item recheck condition (unnecessary here)
  \end{itemize}
\end{frame}

\begin{frame}
  \frametitle{Read path with AND}
  Select rows whose array column contains elements \sqlinline{1} \emph{and}
  \sqlinline{3}:

  \sqlinline{SELECT * FROM arrs WHERE a @> ARRAY[3, 1, 1, 3];}
  \pause

  \begin{itemize}
    \item extract scan entries: \sqlinline{1}, \sqlinline{3}
      \pause
    \item \textbf{split into required and additional entries}:
      \pause
      \begin{itemize}
        \item ideally, the less frequent scan entry is chosen required:
          \sqlinline{1} should be required
          \pause
        \item currently, hueristics aren't present, so it's arbitrary:
          \sqlinline{1} is additional, \sqlinline{3} is required
      \end{itemize}
      \pause
    \item scan index
      \begin{itemize}
        \item bind: \textinline{a}$'$ $=$ \sqlinline{3}
        \item target: \cinline{YBIdxBaseTupleIdAttributeNumber}
          \pause
        \item get: \sqlinline{'i'}, \sqlinline{'j'}, \sqlinline{'k'}
      \end{itemize}
  \end{itemize}
\end{frame}

\begin{frame}
  \frametitle{Read path with AND (continued)}
  \begin{itemize}
    \item scan main table
      \begin{itemize}
        \item bind: \textinline{pk} $=$ \sqlinline{'i'}
        \item target: (same as before)
          \pause
        \item get: \{\sqlinline{'i'}, \sqlinline{ARRAY[1, 3, 5, 3]}\}
      \end{itemize}
      \pause
    \item scan main table
      \begin{itemize}
        \item bind: \textinline{pk} $=$ \sqlinline{'j'}
        \item target: (same as before)
        \item get: \{\sqlinline{'j'}, \sqlinline{ARRAY[1, 3, 2]}\}
      \end{itemize}
      \pause
    \item scan main table
      \begin{itemize}
        \item bind: \textinline{pk} $=$ \sqlinline{'k'}
        \item target: (same as before)
        \item get: \{\sqlinline{'k'}, \sqlinline{ARRAY[5, 3]}\}
      \end{itemize}
      \pause
    \item recheck condition
      \pause
      \begin{itemize}
        \item throw away row with pk \sqlinline{'k'}
      \end{itemize}
  \end{itemize}
\end{frame}

\begin{frame}
  \frametitle{Read path with OR}
  Select rows whose array column intersects with \sqlinline{ARRAY[1, 2]}:

  \sqlinline{SELECT * FROM arrs WHERE a && ARRAY[2, 1, 2];}
  \pause

  \begin{itemize}
    \item extract scan entries: \sqlinline{1}, \sqlinline{2}
      \pause
    \item split into required and additional entries: both are required
      \pause
    \item next, a couple options:
      \pause
      \begin{enumerate}
        \item don't allow more than one required scan entry: user can rewrite
          the query as \sqlinline{WHERE a && ARRAY[1] or a && ARRAY[2]}
          \pause
        \item do point index scans separately, deduplicate pkeys in postgres,
          then do table scans
          \pause
        \item try IN bind: bind \textinline{a}$'$ IN \{\sqlinline{1},
          \sqlinline{2}\}, but make sure to deduplicate pkeys (e.g.\ the row
          with pkey \sqlinline{'j'})
          \pause
        \item create a new bind operator to achieve the above
      \end{enumerate}
  \end{itemize}
\end{frame}

\begin{frame}
  \frametitle{Other items}
  \begin{itemize}
    \item handle nulls
      \pause
    \item implement online and not online index build
      \pause
    \item implement DELETE
      \pause
    \item implement UPDATE
      \pause
    \item handle multicolumn
  \end{itemize}
\end{frame}
